%◆注意事項(本文全体)◆
%
%タイトルに現れる単語・用語をすべて本文中で解説
%略語は,本文中の最初に登場した個所に正式名称を記載 例: IP (Internet Protocol)
%式,方法,システム,技術,アプリケーション,アーキテクチャという用語は,一貫して使用
%不明確な副詞(例: ある程度,かなり,多少,非常に,とても など)は使わず,定量的な表現
%自分が出した結果(結論)は主体的な表現とすべき
%図表の位置はページ中のトップあるいはボトム
%従来方式,関連研究と比較して,自分の研究結果の利点を明示


%◆注意事項(佐藤先生から)◆
%論文に記載する「参考文献」は,著者が論文を執筆する際に参考した文献を意味しているのではありません.
%読者がこの論文を読む際に,論文では十分説明できないので,ここを参考にしてほしいというものです.
%著者の参考ではなく,読者の参考になるものです.
%
%卒論は研究論文であって,解説論文ではありませんので,一般的な技術を丁寧に説明する必要はありません.自分の研究の★内容★に関わる部分のみ,★必要に応じて★一般技術の説明を入れて下さい.自分の研究と関係ない部分は説明する必要はありません.
%
%図は表について,○○に示すだけではなく,その図や表をどのように解釈したらよいのか,内容について説明してください.
%何かの説明を「図に示す」「表に示す」「式に示す」だけではなく,その図,表,式をどう解釈して,どう読み取るのかを文章で記載して下さい.
%
%研究業績は,参考文献と同じように,雑誌名,ページ数,発行年の形式にして下さい.
%
%,「...したい」と記載するのではなく,「◯◯するためには,さらに◯◯が必要である.」みたいな感じがよいと思います.

%図のフォントは,(図全体を小さくして)本文のフォントと同じぐらいの大きさがよいと思います(極端に文字が大きいとカッコ悪い)

\documentclass[a4paper,11pt]{ujreport}
\usepackage{./style/graduatethesis}
\usepackage{listings,./style/jlisting}

\newcommand{\Tabref}[1]{表~\ref{#1}}
\newcommand{\Equref}[1]{式~(\ref{#1})}
\newcommand{\Figref}[1]{図~\ref{#1}}
\newcommand{\Chapref}[1]{第~\ref{#1}章}
\newcommand{\Coderef}[1]{ソースコード~\ref{#1}}

\title{タイトル}

\author{名前}
\department
{情報システムデザイン学科}
\date
{2018年2月20日}
\advisor
{佐藤 健哉 教授}
\entranceyear
{2014}
\registernumber
{30}

\begin{document}
\maketitle
\begin{abstract}

%============================ 概要 ===================================
%研究結果および論文の結論まで含めて書く
%概要の中では参考文献を引用しないのが一般的

\addkeywords{キー}{わー}{ど}
\vspace{11cm}
\renewcommand{\thefootnote}{\fnsymbol{footnote}}

\footnote[0]{本論文に掲載の製品名・会社名等は,一般にそれぞれの会社の商標,または登録商標である.}
\footnote[0]{なお,本文中では\texttrademark ・ \textregistered 等のマークは特に明記していない.}
\end{abstract}


%----------------------------------------------------------------------
%目次作成部分(変更不要)

%ページ番号をギリシャ数字にする
\pagenumbering{roman}
%目次を1ページから始めるために表紙を0ページにする
\setcounter{page}{0}

%目次を作成
\tableofcontents
%改ページ
\newpage
%以降をアラビア数字で振り直します.
\pagenumbering{arabic}
%----------------------------------------------------------------------

%============================ 第1章 ===================================
\chapter{はじめに}\label{chap:beginning}
%なぜこの研究をする必要があるのか,一般的な世間の状況と,研究を行う必要性
はじめに,本研究の背景,目的,構成を述べる.

%本論文の構成について述べる.
%----------------------------------------------------------------------

%◆ 背景と目的
%
% なぜ,この研究をする必要があるのか,一般的な世間の状況と,研究を行う必要性を書く
% 世の中でその問題にどう取り組んでいるかは,一般には「関連研究」のところで説明
% 一般的な背景情報はこの章にまとめる(本文中に一般的背景情報を書くべきではない)
% 論文の技術的な内容や結果を書く必要がない

\section{背景}

%----------------------------------------------------------------------
\section{目的}

%----------------------------------------------------------------------
\section{本論文の構成}

%============================ 第2章 ===================================
\chapter{関連手法}\label{chap:relative_method}

本章では, 関連手法について述べる.

%============================ 第3章 ===================================
\chapter{提案手法}\label{chap:prpposal}

%     問題に関して,自分の解決方法を説明
%     問題そのものを簡単に理解できない場合は,問題についても詳解が必要
%     問題をどうやって解決するかを手順を追って解説
%     複数の問題が関連している場合は問題を分離して説明
%     自分の解決方法が従来とどこが違いどう工夫しているかを明記する必要

本章では,提案手法の概要とその動作について述べる.

%============================ 第4章 ===================================
\chapter{評価}\label{chap:evaluation}
%    自分の解決方法を問題点に適応してどういう結果が得られたかについて説明
%    従来技術や手法と比較してどこがどうよくなったかを示す
%    どのような環境で比較したかを説明
%    定量的に以前(関連研究)と比べてこうよくなったと説明

本章では,提案手法の検証実験に関する評価方法とそれによって得られた評価結果について述べる.

%============================ 第5章 ===================================
\chapter{考察}\label{chap:discussion}
%なるべく考察を章で分ける.
%提案方式に関する「評価」に関して記載する章があるのが一般的かと思いますが,「考察」の章では,従来技術や関連研究と比較して,評価結果がどうであったかを考察して下さい.


%============================ 第6章 ===================================
\chapter{まとめ}\label{chap:end_chapter}
%    * 800 字から1000字ぐらいでまとめてください.(卒論は気にしなくて良い)
%    * 背景,解決すべき問題点,提案内容,結果,考察,研究の意義などを含めて記載してください.
%    * 結果については,過去形(・・・実施した.・・・評価した.・・・確認した.など)で記載してください.
%    背景,解決すべき問題点,提案内容,結果,考察,研究の意義がすべて含まれているか
%検討項目や今後の課題は「考察」に記載し,「まとめ」に記載しないこと)

%============================ 謝辞 ===================================
\chapter*{謝辞}
\thispagestyle{empty}

%============================ 参考文献 ==============================
%    * 勉強した書籍を列挙するものではない    .
%    * 本文中に引用した技術などを記載
%    * 参考文献の番号は,必ず本文中の引用場所を示す
%著者,タイトル,出典,年号の形式は例の通りに正確に記載
%(例:周 劼, 綾木 良太, 島田 秀樹, 佐藤 健哉, クラウドサービスにおける分散コンポーネントフレームワークの提案, 情報処理学会論文誌, Vol.52, No.2, pp.415-423, 2011.)
%参考文献は,URLなどの参照以外に,★論文★が5以上
%単に勉強のために参考にした書籍は列挙する必要なし.


\begin{thebibliography}{99}
\thispagestyle{empty}

\bibitem{ronbun1}名前,"タイトル",情報管理,Vol.60,No.4,pp.229-239,2017.


\end{thebibliography}

\end{document}
